\chapterimage{chapter-bg.png} % Chapter heading image

\chapter{Introduction}

\emph{Stellarium} is a software project that allows people to use their
home computer as a virtual planetarium. It calculates the positions of
the Sun and Moon, planets and stars, and draws how the sky would look to
an observer depending on their location and the time. It can also draw
the constellations and simulate astronomical phenomena such as meteor
showers, and solar or lunar eclipses.

Stellarium may be used as an educational tool for teaching about the
night sky, as an observational aid for amateur astronomers wishing to
plan a night's observing, or simply as a curiosity (it's fun!). Because
of the high quality of the graphics that Stellarium produces, it is used
in some real planetarium projector products. Some amateur astronomy
groups use it to create sky maps for describing regions of the sky in
articles for newsletters and magazines.

The development of a powerful scripting system has been continuing for a
number of years now and can now be called operational. The use of a
script was recognised as a perfect way of arranging a display of a
sequence of astronomical events from the earliest versions of Stellarium
and a simple system called the Stratoscript was implemented. The
scipting facility is Stellarium's version of a \emph{Presentation}, a
feature that may be used to run an astronomical or other presentation
for instruction or entertainment from within the Stellarium program. The
original Stratoscript was quite limited in what it could do so a new
Stellarium Scripting System has been developed.

Stellarium is under fairly rapid development, and by the time you read
this guide, a newer version may have been released with even more
features that those documented here. Check for updates to Stellarium at
the Stellarium website.

If you have questions and/or comments about this guide, please email the
author. For comments about Stellarium itself, visit the
\href{https://sourceforge.net/p/stellarium/discussion/278769/}{Stellarium
forums}.
