% Status info:
% M. Gates	2006-2009
% A. Wolf	2011-2015
% ArdWar	2012
% B. Gerdes	2013
% Additions inserted from wiki 2015-12-26
% Content OK for 0.14+.
% TODO: add historical notes
% TODO: typo&grammar check


\chapter{Introduction}
\label{ch:Introduction}

\emph{Stellarium} is a software project that allows people to use their
home computer as a virtual planetarium. It calculates the positions of
the Sun and Moon, planets and stars, and draws how the sky would look to
an observer depending on their location and the time. It can also draw
the constellations and simulate astronomical phenomena such as meteor
showers, and solar or lunar eclipses.

Stellarium may be used as an educational tool for teaching about the
night sky, as an observational aid for amateur astronomers wishing to
plan a night's observing, or simply as a curiosity (it's fun!). Because
of the high quality of the graphics that Stellarium produces, it is used
in some real planetarium projector products. Some amateur astronomy
groups use it to create sky maps for describing regions of the sky in
articles for newsletters and magazines.

The development of a powerful scripting system has been continuing for
a number of years now and can now be called operational. The use of a
script was recognised as a perfect way of arranging a display of a
sequence of astronomical events from the earliest versions of
Stellarium and a simple system called \emph{Stratoscript} was
implemented. The scipting facility is Stellarium's version of a
\emph{Presentation}, a feature that may be used to run an astronomical
or other presentation for instruction or entertainment from within the
Stellarium program. The original \emph{Stratoscript} was quite limited in
what it could do so a new Stellarium Scripting System has been
developed. As of version 0.14.0 a new scripting engine 
has reached a level where it 
has all required features for usage, however new
commands may be added from time to time. Since version 0.14.0 support
of scripts for the \emph{Stratoscript} engine has been discontinued.

Stellarium is still under  development, and by the time you read
this guide, a newer version may have been released with even more
features that those documented here. Check for updates to Stellarium at
the Stellarium website.

If you have questions and/or comments about this guide, 
or about Stellarium itself, visit the Stellarium forums\footnote{
\url{https://sourceforge.net/p/stellarium/discussion/278769/}}.


\section{Historical notes}
\label{sec:Introduction:HistoricalNotes}

%% GZ: I put meta-comments under %% and outlines of things which should be included but which I cannot write under single comment signs (%).

%% TODO: A short history. Please one of the original team, this would be really useful to fill in!!!
%% It may also be possible, if everyone from the original team adds 1/2-1 page under his name, to have some nice anecdotes?

Stellarium has been developed by a small team of enthusiastic developers. 

Fabien Chereau started the project during the summer 2000, and throughout
the years found continuous support by a small team of enthusiastic developers.
Unfortunately time is evolving, and most members of the original
development team are no longer able to devote most of their spare time
to the project (some are still available for limited work
which requires specific knowledge about the project).

Here is a list of past and present major contributors sorted roughly by date of
arrival on the project:
 - Fabien Chereau: original creator, maintainer, general development
 - Matthew Gates: maintainer, user guide, user support, general development
 - Johannes Gajdosik: astronomical computations, large stars catalogs support
 - Johan Meuris: GUI design, website creation, drawings of our Western 88 constellations
 - Nigel Kerr: Mac OSX port
 - Rob Spearman: funding for planetarium support
 - Barry Gerdes: user support, tester, windows support. Barry passed away in 
     October 2014 at age 80. He was a major contributor on the forums, wiki
     pages and mailing list where his good will and enthusiasm is strongly
     missed. RIP Barry.
 - Timothy Reaves: ocular plugin
 - Bogdan Marinov: GUI, telescope control, other plugins
 - Diego Marcos: SVMT plugin
 - Guillaume Chereau: display, optimization, Qt upgrades
 - Alexander Wolf: maintainer, general development
 - Georg Zotti: astronomical computations, user support
 - Marcos Cardinot: meteor plugin
 - Florian Schaukowitsch: 3D Landscape plugin

A few important milestones for the project:
 - 2000: first lines of code for the project
 - 2001-06: first public mention (and users feedbacks!) of the software on the fr.sci.astronomie.amateur french newsgroup \url{https://groups.google.com/d/topic/fr.sci.astronomie.amateur/OT7K8yogRlI/discussion}
 - 2003-01: stellarium reviewed by Astronomy magazine
 - 2003-07: funding for developing planetarium features (fisheye projection and other features)
 - 2005-12: use accurate (and fast) planetary model
 - 2006-05: stellarium "Project Of the Month" on sourceforge
 - 2006-08: large stars catalogs
 - 2007-01: funding by ESO for development of professional astronomy extensions (VirGO)
 - 2007-04: developer's meeting near Munich, Germany
 - 2007-05: switch to the Qt library as main GUI and general purpose library
 - 2009-09: plugin system, enabling a lot of new development
 - 2010-07: stellarium ported on Maemo mobile device
 - 2010-11: artificial satellites plugin
 - 2014-06: high quality satellites and saturn rings shadows, moon craters
 - 2014-07: adapt to opengl evolutions in the Qt framework, require more modern graphic hardware than earlier versions
 - 2015-04: scenery 3D plugin

As of 2016, the project's maintainer is Alexander Wolf, doing most maintenance
and regular releases. Major new features are contributed mostly by Georg Zotti
and his team focused on extensions of Stellarium's applicability in the fields of
historical and cultural astronomy (which means Stellarium is getting more
accurate), but also on graphic items like the comet tails or the Zodiacal Light.


Stellarium has been kindly supported by ESA in their Summer of Code in
Space initiatives, which resulted in better planetary rendering
(2012), and the web-based remote control and an alternative solution
for planetary positions based on the DE430/DE431 ephemeris (2015).


This guide is based on the user guide written by Matthew Gates for
version 0.10 around 2008, and updated by Barry Gerdes up to version
0.12. %% PLEASE CORRECT ME!
The user documentation has been developed on the Stellarium wiki for
some time, but we (Alexander and Georg) feel that a single book may
be the better format for offline reading. The PDF version of this 
guide has a clickable table of contents and clickable hyperlinks.

This new edition of the guide will not contain notes about using
earlier versions than 0.13 or using very outdated hardware. Some
references to previous version may still be made for completeness, 
but if you are using earlier versions
for particular reasons, please use the older guides.

%%% Local Variables: 
%%% mode: latex
%%% TeX-master: "guide"
%%% End: 
