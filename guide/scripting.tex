
\chapter{Scripting}
\label{ch:scripting}

Many functions in Stellarium are scriptable. The programming language
ECMAscript\footnote{\url{https://en.wikipedia.org/wiki/ECMAScript}} (also known as JavaScript) can be used to control most
aspects of the software to construct automated shows.

\section{Introduction}
\label{sec:scripting:introduction}

Since version 0.10.1, Stellarium includes a scripting feature based on the Qt Scripting Engine\footnote{\url{http://doc.qt.io/qt-5/qtscript-index.html}}. This makes it possible to write small programs within Stellarium to produce presentations, set up custom configurations, and to automate repetitive tasks. 

The core scripting language is ECMAscript, giving users access to all basic ECMAScript language features such as flow control, variables string manipulation and so on. Interaction with Stellarium-specific features is done via a collection of objects which represent components of Stellarium itself. See appendix~\ref{ch:ScriptingAPI} for more details.

\section{Includes}
\label{sec:scripting:includes}

Stellarium provides mechanism for splitting scripts on different files. Typical functions or list of variables can be stored in separate .inc file and used within script through \textbf{include()} command:
\begin{script}
include("common_objects.inc");
\end{script}

\section{Script Console}
\label{sec:scripting:console}
It is possible to open, edit run and save scripts using the script console window. To toggle the script console, press \key{F12}. The script console also provides an output window in which script debugging output is visible.

\section{Examples}
\label{sec:scripting:examples}
The best source of examples is the \textbf{scripts} sub-directory of the main Stellarium source tree. This directory contains a sub-directory called \textbf{tests} which are not installed with Stellarium, but are nontheless useful sources of example code for various scripting features\footnote{The directory can be browsed online at \url{http://bazaar.launchpad.net/~stellarium/stellarium/trunk/files/head:/scripts/}. Script files end in .ssc and .inc. Download links are to the right.}.

\subsection{Minimal Script}
\label{sec:scripting:MinimalScript}
This script prints "Hello Universe" in the Script Console output window.
\begin{script}
core.debug("Hello Universe");
\end{script}

\subsection{Using a StelModule}
\label{sec:scripting:UsingStelModule}
This script uses the LabelMgr module to display "Hello Universe" in white on the screen for 3 seconds.
\begin{script}
LabelMgr.labelScreen("Hello Universe", 200, 200, true, 20, "#ff0000");
core.wait(3);
LabelMgr.deleteAllLabels();
\end{script}

%% TODO: copy most of the Scripting API docs here.