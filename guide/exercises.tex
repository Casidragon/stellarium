\chapterimage{chapter-t4-bg} % Chapter heading image

\chapter{Exercises}

\section{Find M31 in Binoculars}\label{find-m31-in-binoculars}

M31 -- the Andromeda Galaxy -- is the most distant object visible to the
naked eye. Finding it in binoculars is a rewarding experience for
new-comers to observing.

\subsection{Simulation}\label{simulation}

\begin{enumerate}
\item
  Set the location to a mid-Northern latitude if necessary (M31 isn't
  always visible for Southern hemisphere observers). The UK is ideal.
\item
  Find M31 and set the time so that the sky is dark enough to see it.
  The best time of year for this at Northern latitudes is Autumn/Winter,
  although there should be a chance to see it at some time of night
  throughout the year.
\item
  Set the field of view to $6\degree$ (or the field of view of your binoculars
  if they're different. $6\degree$ is typical for 7x50 bins).
\item
  Practise finding M31 from the bright stars in Cassiopeia and the
  constellation of Andromeda.
\end{enumerate}

\subsection{For Real}\label{for-real}

This part is not going to be possible for many people. First, you need a
good night and a dark sky. In urban areas with a lot of light pollution
it's going to be very hard to see Andromeda.

\section{Handy Angles}\label{handy-angles}

As described in section \href{Exercises\#Handy_Angles}{Handy Angles},
your hand at arm's length provides a few useful estimates for angular
size. It's useful to know if your handy angles are typical, and if not,
what they are. The method here below is just one way to do it -- feel
free to use another method of your own construction!

Hold your hand at arm's length with your hand open -- the tips of your
thumb and little finger as far apart as you can comfortably hold them.
Get a friend to measure the distance between your thumb and your eye,
we'll call this \emph{D}. There is a tendency to over-stretch the arm
when someone is measuring it -- try to keep the thumb-eye distance as it
would be if you were looking at some distant object.

Without changing the shape of your hand, measure the distance between
the tips of your thumb and little finger. It's probably easiest to mark
their positions on a piece of paper and measure the distance between the
marks, we'll call this d. Using some simple trigonometry, we can
estimate the angular distance $\theta$:

Repeat the process for the distance across a closed fist, three fingers
and the tip of the little finger.

For example, for the author \emph{D}=72 cm, \emph{d}=21 cm, so:

\begin{equation}
\theta = 2 \cdot \arctan{\left( \dfrac{21}{144} \right)} \approx 16\frac{1}{2}\degree
\end{equation}

Remember that handy angles are not very precise -- depending on your
posture at a given time the values may vary by a fair bit.

\section{Find a Lunar Eclipse}\label{find-a-lunar-eclipse}

Stellarium comes with two scripts for finding lunar eclipses, but can
you find one on a different date?

\section{Find a Solar Eclipse}\label{find-a-solar-eclipse}

Find a Solar Eclipse using Stellarium \& take a screenshot of it.