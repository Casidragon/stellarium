%% Status: 
%% AW from wiki or old guide? 2015-12-27
%% GZ 2016-01-11 typofixes, relabeled sections, added index entries
%% OK for 0.13+
%% Suggestions: Extend sections on eclipses.

\chapter{Exercises}
\label{ch:Exercises}

\section{Find M31 in Binoculars}
\label{sec:Exercises:M31}

M31\index{M31} -- the \indexterm{Andromeda Galaxy} -- is
the most distant object visible to the naked eye. Finding it in
binoculars\index{binoculars} is a rewarding experience for new-comers to observing.

\subsection{Simulation}

\begin{enumerate}
\item
  Set the location to a mid-Northern latitude if necessary (M31 isn't
  always visible for Southern hemisphere observers). The UK is ideal.
\item Find M31 and set the time so that the sky is dark enough to see
  it.  The best time of year for this at Northern latitudes is
  Autumn/Winter, although there should be a chance to see it at some
  time of night throughout the year.
\item Set the field of view to $6\degree$ (or the field of view of
  your binoculars if they're different. $6\degree$ is typical for 7x50
  binoculars).
\item Practise finding M31 from the bright stars in Cassiopeia and the
  constellation of Andromeda. Learn the chain of stars that extends
  from Andromeda's central star perpendicular to her body.
\end{enumerate}

\subsection{For Real}

This part is not going to be possible for many people. First, you need a
good night and a dark sky. In urban areas with a lot of light pollution
it's going to be very hard to see Andromeda.

\section{Handy Angles}
\label{sec:Exercises:handyAngles}
\index{Handy Angles}


As described in section~\ref{sec:Concepts:Angles:HandyAngles}, 
your hand at arm's length provides a few useful estimates for angular
size. It's useful to know whether your handy angles are typical, and if not,
what they are. The method here below is just one way to do it -- feel
free to use another method of your own construction!

Hold your hand at arm's length with your hand open -- the tips of your
thumb and little finger as far apart as you can comfortably hold them.
Get a friend to measure the distance between your thumb and your eye,
we'll call this $D$. There is a tendency to over-stretch the arm
when someone is measuring it -- try to keep the thumb-eye distance as it
would be if you were looking at some distant object.

Without changing the shape of your hand, measure the distance between
the tips of your thumb and little finger. It's probably easiest to mark
their positions on a piece of paper and measure the distance between the
marks, we'll call this $d$. Using some simple trigonometry, we can
estimate the angular distance $\theta$ using equation~\eqref{eq:handyAngle}.

Repeat the process for the distance across a closed fist, three fingers
and the tip of the little finger.

For example, for one author $D=72\cm$, $d=21\cm$, so:

\begin{equation}
\theta = 2 \cdot \arctan{\left( \frac{21}{144} \right)} \approx 16 \frac{1}{2}\degree
\end{equation}

Remember that handy angles are not very precise -- depending on your
posture at a given time the values may vary by a fair bit.

\section{Find a Lunar Eclipse}
\label{sec:Exercises:LunarEclipse}

Stellarium comes with two scripts for finding lunar eclipses, but can
you find one on a different date?
%% TODO: check scripts for more? Name those scripts?

\section{Find a Solar Eclipse}
\label{sec:Exercises:SolarEclipse}

Find a Solar Eclipse using Stellarium and take a screenshot of it. Use
the location panel and see how the eclipse look on different locations
at the same time.



%%% Local Variables: 
%%% mode: latex
%%% TeX-PDF-mode: t
%%% TeX-master: "guide"
%%% End: 
